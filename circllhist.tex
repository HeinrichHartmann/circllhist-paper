\documentclass{article}

\usepackage{arxiv}

\usepackage[utf8]{inputenc} % allow utf-8 input
\usepackage[T1]{fontenc}    % use 8-bit T1 fonts
%\usepackage{hyperref}       % hyperlinks
\usepackage{url}            % simple URL typesetting
\usepackage{booktabs}       % professional-quality tables
\usepackage{amsfonts}       % blackboard math symbols
\usepackage{nicefrac}       % compact symbols for 1/2, etc.
\usepackage{microtype}      % microtypography
\usepackage{minted}
\title{circllhist}

\subtitle{The Circonus Log-Linear Histogram}

\author{
  Heinrich Hartmann \\
  \texttt{heinrich.hartmann@circonus.com} \\
  Circonus \\
  \And
  Theo Schlossnagle \\
  \texttt{theo.schlossnagle@circonus.com} \\
  Circonus
}

\begin{document}
\maketitle

\begin{abstract}
  The circllhist histogram is a simple, fast and memory efficient data structure for capturing
  and processing large number of samples, that is particularly suited for applications in
  IT infrastructure monitoring.

  The circllhist allows arbitrary merging of pre-aggregated data without additional loss of accuracy,
  and the approximation of percentiles with low expected error and a-priori bounded maximal error.

  Open-source implementations are available for C/lua/python/Go/Java/JavaScript.
\end{abstract}

\tableofcontents

\section{Introduction}
The circllhist is a histogram data structure that allows the representation of an virtually
unlimited amount of data with bounded memory consumption, and precise a-priori bounds for the
accuracy of derived statistics.

In this note we will describe the data structure ...

\section{Related Work}

\begin{itemize}
\item HdrHistogram
\item t-digest
\end{itemize}

\section{The Circllhist Datastructure}
\begin{minted}[frame=lines,framesep=2mm]{c}

struct histogram {
  uint16_t allocd;
  uint16_t used;
  struct hist_bv_pair *bvs;
};

struct hist_bv_pair {
  hist_bucket_t bucket;
  uint64_t count;
};

typedef struct hist_bucket {
  int8_t val;
  int8_t exp;
} hist_bucket_t;
\end{minted}

\subsection{Circllhist as HdrHistogram}

\section{Histogram Operations}

\subsection{Merging}

\subsection{Mean Values}

\subsection{Percentiles}

\section{Evaluation: Performance}

\section{Evaluation: Accuracy}


\bibliographystyle{unsrt}
\begin{thebibliography}{1}

\bibitem{t-digest}
George Kour and Raid Saabne.
\newblock Real-time segmentation of on-line handwritten arabic script.
\newblock In {\em Frontiers in Handwriting Recognition (ICFHR), 2014 14th
  International Conference on}, pages 417--422. IEEE, 2014.

\bibitem{HdrHistogram}
George Kour and Raid Saabne.
\newblock Real-time segmentation of on-line handwritten arabic script.
\newblock In {\em Frontiers in Handwriting Recognition (ICFHR), 2014 14th
  International Conference on}, pages 417--422. IEEE, 2014.

\bibitem{kour2014real}
George Kour and Raid Saabne.
\newblock Real-time segmentation of on-line handwritten arabic script.
\newblock In {\em Frontiers in Handwriting Recognition (ICFHR), 2014 14th
  International Conference on}, pages 417--422. IEEE, 2014.

\end{thebibliography}
\end{document}
